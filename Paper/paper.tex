\documentclass{llncs}
\usepackage[ngerman]{babel}
\usepackage[utf8]{inputenc}
\usepackage[T1]{fontenc}
\begin{document}

\pagestyle{headings}

\mainmatter

\title{Adaptive Gestenerkennung mit Variationsabschätzung für interaktive Systeme}

\titlerunning{Gestenerkennung}

\author{Maxim Boianetchii\inst{1} \and Marian Stein\inst{2}}

\authorrunning{Maxim Boianetchii,  Marian Stein}

\institute{Universität Rostock\\
\email{maxim.boianetchii@uni-rostock.de}
\and
Universität Rostock\\
\email{marian.stein@uni-rostock.de}}

\maketitle
\section{Einführung}


\section{Ähnliche Arbeiten}


\section{Interaktionsprinzipien}


\section{Zugrunde liegendes Modell}


\section{Erkennung von realen 2D-Gesten}
Um die Gestenauswertung zu evaluieren, wurde die Gestendatenbank von Wobbrock et al.\cite{Wobbrock2007} verwendet. Diese enthält Daten von 16 Stiftgesten, die von zehn Teilnehmern in drei verschiedenen Geschwindigkeiten jeweils zehn mal aufgenommen wurden. Für die Versuche wurden pro Geste zufällig aus den Daten eines Teilnehmers jeweils ein Trainings- und ein anderer Testdatensatz ausgewählt. Insgesamt wurden 4 verschiedene Tests durchgeführt, die jeweils 100 Mal wiederholt wurden:
\begin{enumerate}
\item Gleiche Testbedingungen, wie sie von Wobbrock\cite{Wobbrock2007} verwendet wurden.
\item Einfluss von geänderten Verteilungsparametern.
\item Verwendung von Trainings- und Testdaten mit unterschiedlichen Geschwindigkeiten.
\item Genauigkeit der Erkennung von Gesten, während sie noch nicht abgeschlossen sind.
\end{enumerate}
Da GVF auch Veränderungen in den Gesten erkennen und ausgeben soll, ergibt sich für diese Tests ein Zustandsraum $x_k$, der aus der Phase $p_k$, der Geschwindigkeit $v_k$, der Skalierung $s_k$ und dem Drehwinkel $\alpha_k$ besteht.



\bibliography{paper}
\bibliographystyle{plain}
\end{document}
